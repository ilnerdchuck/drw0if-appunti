\section{Crittografia a chiave pubblica}
Ricorcdando il one-time-pad il suo più grande problema risulta essere la lunga chiave, che va scambiata e ricreata ogni volta.
Anche AES necessita di scambio di chiavi.
\emph{Come si può scambiare una chiave in maniera sicura?}
Nel 1976 Diffie ed Hellman con il loro articolo \emph{New Directions in Cryptography} hanno rivoluzionato la crittografia.
Il protocollo \emph{DH} permette la negoziazione di una chiave di sessione senza che le due parti si siano scambiate informazioni in precedenza.
In parallelo sempre gli stessi Diffie ed Hellman propongonono lo schema di crittografia a chiave pubblica, senza tuttavia averne una valida implementazione.
In questo nuovo schema si hanno due chiavi:
\begin{itemize}
    \item una \emph{pubblica} nota a tutti ed utilizzata per cifrare
    \item una \emph{privata} nota solo al ricevente usata per decifrare
\end{itemize}
Ogni utente avrà quindi una coppia
$$ <K_{priv}, K_{pub}> $$

NB: su $n$ utenti ci saranno quindi $n$ coppie di chiavi contro le $n^2$ chiavi nel caso di algoritmi simmetrici

La cifratura ha la forma:
$$ c = C(m, K_{pub}) $$
mentre la decifrazione ha la forma:
$$ m = D(c, K_{priv})$$

NB: si dice asimmetrica perché le due parti fanno cose diverse, non c'è una simmetria.

Questi cifrari devono avere alcune caratteristiche:
\begin{itemize}
    \item $D(C(m, K_{pub}), K_{priv}) = m$
    \item efficienza e sicurezza: le chiavi devono essere \emph{facili} da generare e praticamente deve essere impossibile generare la stessa coppia.
    Ciò che è legale deve essere fatto in tempo polinomiale, mentre deve essere difficile ottenere il messaggio conoscendo $c$ e $K_{pub}$
\end{itemize}

Questi requisiti si hanno se si usano le \emph{funzioni one-way trap-door}, funzioni facili da calcolare ma difficili da invertire senza avere maggiori informazioni.

Nel 1977 Rivest, Shamir ed Adleman scoprono una funzione di questo tipo basata sui numeri primi.
L'algoritmo RSA si basa sul fatto che dati $p$ e $q$ primi, calcolare $n = p \cdot q$ è facile, dato $n$ invece è difficile trovate $p$ e $q$.
Inoltre lavora in algebra modulare!

\subsection{Algebra modulare}
E' fondamentale in crittografia in quanto:
\begin{itemize}
    \item riduce lo spazio dei numeri sui quali si opera e rende più veloci i calcoli
    \item rende difficili alcuni problemi computazionalmente semplici (o banali) nell'algebra standard
\end{itemize}

Nell'algebra modulare ad esempio le funzioni tendono ad essere imprevedibili:
\begin{table}[ht!]
    \centering
    \small
    \begin{tabular}{c|c c c c c c c c c c c c}
        $x$ & 1 & 2 & 3 & 4 & 5 & 6 & 7 & 8 & 9 & 10 & 11 & 12  \\
        \hline
        $2^{x}$ & 2 & 4 & 8 & 16 & 32 & 64 & 128 & 256 & 512 & 1024 & 2048 & 4096  \\
        $2^{x} \mod 13$ & 2 & 4 & 8 & 3 & 6 & 12 & 11 & 9 & 5 & 10 & 7 & 1  \\
    \end{tabular}
\end{table}

Mentre nell'algebra normale la funzione esponenziale è monotòna, nell' algebra modulare la funzione è imprevedibile.

Dato $n \in \mathbb{N}$ allora:
$$ Z_{n} = \{ 0, 1, 2, \_, n-1 \} $$
$$ Z_{n}^{*} \subseteq Z_{n} = \{\text{elementi di $Z_{n}$ coprimi con $n$}\} $$


NB: Se $n$ è primo: $Z_{n}^* = \{1, 2, 3, \_, n-1\}$

Dati $a, b \geq 0$ interi e $n > 0$ diremo \emph{a è congruo a b modulo n}:
$$ a \equiv b \mod n $$
se e solo se $\exists k$ intero per cui:
$$ a = b + k \cdot n $$
Quindi
$$ a \equiv b \mod n \iff a \mod n = b \mod n $$

NB: $a \mod n = b \mod n$, nella relazione di congruenza mod n si riferisce all'intera relazione, nelle relazioni di uguaglianza invece si riferisce solo al membro dove appare:
$$ 5 \neq 8 \mod 3 $$
$$ 2 = 8 \mod 3 $$

\subsubsection{Altre proprietà}
$$ (a+b) \mod m = (a \mod m + b \mod m) \mod m $$
$$ (a-b) \mod m = (a \mod m - b \mod m) \mod m $$
$$ (a \cdot b) \mod m = (a \mod m \cdot b \mod m) \mod m $$
$$ a^{r \times s} \mod m = (a^r \mod m)^{s} \mod m $$

\subsection{Funzione di Eulero}
$$ \phi(n) = \mid Z_{n}^{*} \mid $$
indica il numero di interi minori di $n$ e coprimi con esso.

NB: $n$ primo $\implies$ $\phi(n) = n-1$

Se $n$ è composto
$$ \phi(n) = n \cdot (1 - \frac{1}{p_1}) \cdot \_ \cdot (1 - \frac{1}{p_k})$$
con $p_1$, \_, $p_k$ i fattori primi di n presi senza molteplicità

Se $n = p \cdot q$ con $p$ e $q$ primi (quindi $n$ semiprimo):
$$ \phi(n) = (p-1) \cdot (q-1) $$

\subsection{Teorema di Eulero}
$\forall n>1$, $\forall a \in Z_{n}^*$:
$$ a^{\phi(n)} \equiv 1 \mod n $$

\subsection{Piccolo teorema di Fermat}
$\forall n$ primo e $\forall a \in Z_{n}^*$:
$$ a^{n-1} \equiv 1 \mod n $$

NB: $ a^{\phi(n)} \equiv 1 \mod n $ ma essendo $n$ primo $\phi(n) = n-1 \implies  a^{n-1} \equiv 1 \mod n $

\subsection{Conseguenze}
$\forall a \in Z_{n}^*$
$$ a^{\phi(n)} = a \cdot a^{\phi(n)-1} \implies a \cdot a^{\phi(n)-1} \equiv 1 \mod n $$
$$ a \cdot a^{-1} \equiv 1 \mod n $$
$$ a^{-1} = a^{\phi(n)-1} \mod n $$
Quindi $a^{-1} \mod n$ si può calcolare se si conosce $\phi(n)$, quindi se si conoscono $p, q, \_$ primi fattori di $n$.

NB: in generale nell'algebra modulare l'esistenza dell'inverso non è garantita perché $a^{-1}$ deve essere intero.

\subsection{Teorema sull'inverso}
L'equazione $a \cdot x \equiv b \mod n$ ammette soluzioni se e solo se $mcd(a,n)$ divide $b$.
In questo caso si hanno esattamente $mcd(a,n)$ soluzioni distinte.

NB: se $mcd(a,n)=1$ ($a$,$n$ coprimi) $a \cdot x \equiv b \mod n$ ammette una unica soluzione. E quindi esiste $a^{-1}$


\subsection{Algoritmo di Euclide esteso}
Nasce per risolvere l'equazione $a \cdot x + b \cdot y = mcd(a,b)$.
\begin{verbatim}
    Euclide_esteso(a,b):
        se b == 0:
            return (a,1,0)
        _d, _x, _y = Euclide_esteso(b, a mod b)
        d, x, y = _d, _y, _x - (a//b) * _y
        return (d, x, y)
\end{verbatim}
Ha complessità logaritmica nel valore di $a$ e $b$ quindi polinomiale nella dimensione dell'input.

Possiamo usare quest'algoritmo per il calcolo dell'inverso:
$$ ax \equiv 1 \mod n $$
$$ ax = nz + 1 \text{ per un certo valore di z} $$
$$ ax + ny = mcd(a,b) \text{ con y = -z, n = b, 1 = mcd(a,b)} $$
usando l'algoritmo si ottengono $<d, x, y>$, $x$ è proprio il nostro inverso.

\subsection{Generatori}
$a \in Z_{n}^*$ è un \emph{generatore di $Z_{n}^*$} se la funzione:

$$ a^k \mod n \text{ , } 1 \leq k \leq \phi(n) $$

genera \emph{tutti e soli} gli elementi di $Z_{n}^*$.

Un generatore quindi genera $Z_{n}^*$ ma in un ordine difficile da prevedere.

NB: l'unica cosa prevedibile è che $1$ si ottiene per $k = \phi(n)$ poiché sappiamo $a^{\phi(n)} \equiv 1 \mod n$ per Eulero. Quindi $\forall a^k$ $\neq 1 \mod n$, $ 1 \leq k < \phi(n)$

Non tutti gli $n$ forniscono generatori per $Z_{n}^*$.

\subsection{Teorema sui generatori}
Se $n$ è primo allora $Z_{n}^*$ ha almeno un generatore.

Per $n$ primo non tutti gli elementi di $Z_n^*$ sono suoi generatori (1 non lo è mai).

Per $n$ primo i generatori di $Z_n^*$ sono $\phi(n)$.

\subsection{Problemi sui generatori}
Determinare un generatore di $Z_n^*$ con $n$ primo è difficile, si possono provare tutti i numeri in $[2, n-1]$ ma è esponenziale nella dimensione di $n$.
Esistono tuttavia algoritmi randomizzati che risolvono il problema efficientemente.

\subsection{Logaritmo discreto}
Consiste nel trovare $x$ che risolve:
$$ a^x \equiv b \mod n, \text{n primo} $$
L'equazione ammette soluzioni $\forall b$ se e solo se $a$ è un generatore di $Z_n^*$.
Tuttavia non possiamo sapere in che ordine sono generati i valori di $Z_n^*$ quindi per trovare $x$ giusto dobbiamo iterare tutti i valori (esponenziale nella dimensione dell'input).

NB: su questa complessità si baserà l'algoritmo Diffie-Hellman.

\subsection{Funzioni one-way trap-door}
$$ n = p \cdot q \text{ p, q primi} $$
è facile.

Dato $n$ trovare $p$ e $q$ richiede tempo (sub)esponenziale. Dato $p$ o $q$ invece si fattorizza velocemente: $q = \frac{n}{p}$, ma anche conoscendo $\phi(n)$ si fattorizza velocemente. E' il problema della \emph{fattorizzazione}

$$ y = x^z \mod s \text{ con x, z, s interi} $$
è facile, si può infatti fare in O($\log_2z$).

Se $s$ non è primo calcolare $x = y^{\frac{1}{z}} \mod s$ richiede tempo esponenziale.
L'estrazione di radice con $s$ non primo è quindi difficile! E' il problema della \emph{radice}

$$y = x^z \mod s $$
è facile.

Trovare $z$ tale che $y = x^z mod s$ dati $x, y, s$ è difficile (ha infatti la stessa complessità della fattorizzazione). E' il problema del \emph{logaritmo discreto}.

\subsection{Vantaggi e svantaggi}
Pro:
\begin{itemize}
    \item meno chiavi
    \item non richiedono lo scambio di chiavi
\end{itemize}
Con:
\begin{itemize}
    \item più lenti della cifratura simmetrica
    \item esposizione ad attacchi chosen plain-text
\end{itemize}

Si può infatti criptare un tot di messaggi $m_1, \_, m_h$ e vedere se sul canale passa uno dei vari $c_1, \_, c_h$. Se compare allora sappiamo il messaggio, se non compare allora sappiamo quale messaggio sicuramente non è.

\subsection{Cifrari ibridi}
Si usa un cifrario a chiave pubblica per scambiare le chiavi AES e successivamente si comunica tramite AES che è veloce. L'attacco chosen plain-text è risolto con chiavi random difficilmente prevedibili.

NB: la chiave pubblica va estratta da un certificato digitale per evitare attacchi man-in-the-middle in cui terzi si spacciano per il destinatario.